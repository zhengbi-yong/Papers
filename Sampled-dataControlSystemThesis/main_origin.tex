%!TEX program = xelatex
% 完整编译: xelatex -> biber/bibtex -> xelatex -> xelatex
\documentclass[lang=cn,a4paper,newtx]{elegantpaper}

\title{课程设计}
\author{雍征彼 \\ 北京理工大学}
\institute{}

% \version{0.11}
\date{\zhdate{2024/11/18}}
\usepackage{fontspec}
\setmainfont{Times New Roman} % 或其他已安装字体

% 本文档命令
\usepackage{array}
\usepackage{amsmath}
% Recommended, but optional, packages for figures and better typesetting:
\usepackage{microtype}
\usepackage{graphicx}
\usepackage{subfigure}
\usepackage{booktabs} % for professional tables

% hyperref makes hyperlinks in the resulting PDF.
% If your build breaks (sometimes temporarily if a hyperlink spans a page)
% please comment out the following usepackage line and replace
% \usepackage{icml2024} with \usepackage[nohyperref]{icml2024} above.
\usepackage{hyperref}


% Attempt to make hyperref and algorithmic work together better:
\newcommand{\theHalgorithm}{\arabic{algorithm}}

% Use the following line for the initial blind version submitted for review:
% \usepackage{icml2024}



% For theorems and such
% \theoremstyle{definition}
% \newtheorem{definition}{Definition}
\usepackage{amsmath}
\usepackage{amssymb}
\usepackage{mathtools}
\usepackage{amsthm}
\newcommand{\ccr}[1]{\makecell{{\color{#1}\rule{1cm}{1cm}}}}
\addbibresource[location=local]{reference.bib} % 参考文献,不要删除

\begin{document}

\maketitle

随着人工智能经验学派的发展,Tricks和排列组合的方法逐渐耗尽,而控制理论在现代人工智能技术中的作用则日益体现,作为一个已经发展了较久的学科,其有很多方法论可以为人工智能所用,虽然不一定能够完全照搬,但是至少提供了珍贵的参考。本课程设计旨在通过复杂采样控制系统的学习,探讨控制理论在人工智能系统设计中的应用与发展。本文以一个案例分析,展示了控制理论在Transformer中的应用。

\section{PID与Transformer的结合}
来自Rice大学的Tam Nguyen提出了一种新的Transformer结构,称为PIDformer。PIDformer是在Transformer的基础上,引入了PID控制器,用于对Transformer的输出进行调节。
Transformer架构在表达方面主要存在两个缺陷:输入易受干扰和输出表示秩坍塌。Tam Nguyen揭示了自注意力机制本质上是一种自治的状态空间模型,它自然地在求解中倾向于输出平滑,从而导致输出的秩降低和表示能力的削弱。此外,Transformer自注意力机制的稳态解对输入扰动非常敏感。为了解决这些问题,他在模型中引入了一个带有参考点的比例-积分-微分(PID)闭环反馈控制系统,以提升鲁棒性和表示能力。这种集成方法在增强模型稳定性的同时,保留高频细节信息,使其对噪声更加抗扰。理论证明,这种受控状态空间模型具有较强的鲁棒性,并能够有效解决秩坍塌问题。在这一控制框架的启发下,Tam提出了一种新型Transformer架构——由PID控制的Transformer(PIDformer),旨在提升鲁棒性并缓解Softmax Transformer中固有的秩坍塌问题。



\section{背景知识}
\section{自注意力机制}
\section*{1.1 背景:自注意力机制}

给定一个令牌序列:
\[
\mathbf{X}^{\ell} := 
\begin{bmatrix}
\mathbf{x}^{\ell}(1), \cdots, \mathbf{x}^{\ell}(N)
\end{bmatrix}^\top, \quad 
\mathbf{X}^{\ell} \in \mathbb{R}^{N \times D_x},
\]
在第 $\ell$ 层,查询矩阵、键矩阵和值矩阵分别定义为:
\[
\mathbf{Q}^{\ell} = \mathbf{X}^\ell \mathbf{W}_Q^{\ell^\top}, \quad
\mathbf{K}^{\ell} = \mathbf{X}^\ell \mathbf{W}_K^{\ell^\top}, \quad
\mathbf{V}^{\ell} = \mathbf{X}^\ell \mathbf{W}_V^{\ell^\top},
\]
其中,权重矩阵满足 $\mathbf{W}_Q^{\ell}, \mathbf{W}_K^{\ell} \in \mathbb{R}^{D_{qk} \times D_x}$,$\mathbf{W}_V^{\ell} \in \mathbb{R}^{D \times D_x}$。

自注意力机制计算第 $\ell$ 层中第 $i$ 个令牌的输出如下:
\[
u^{\ell}(i) = \sum_{j=1}^N \text{softmax}\left( 
\frac{\mathbf{q}^{\ell}(i)^\top \mathbf{k}^{\ell}(j)}{\sqrt{D_{qk}}}
\right) \mathbf{v}^{\ell}(j),
\]
其中 $\mathbf{q}^{\ell}(i)$ 是查询矩阵 $\mathbf{Q}^{\ell}$ 的第 $i$ 行,而 $\mathbf{k}^{\ell}(j)$ 和 $\mathbf{v}^{\ell}(j)$ 分别是键矩阵 $\mathbf{K}^{\ell}$ 和值矩阵 $\mathbf{V}^{\ell}$ 的第 $j$ 行。Softmax 函数用于计算令牌 $i$ 和 $j$ 之间的注意力分数。

这种自注意力机制被称为 \textbf{Softmax 注意力}。使用 Softmax 注意力的 Transformer 被称为 \textbf{Softmax Transformer}。

尽管 Transformer 取得了显著的成功,但在鲁棒性和表示能力方面仍存在实际性能问题。例如,近期研究表明,视觉 Transformer 易受到对抗性攻击和常见输入扰动(如噪声或模糊)的影响。此外,深度 Transformer 模型的输出还存在秩坍塌现象,即随着模型深度的增加,令牌嵌入变得越来越相似。这种问题严重限制了 Transformer 的表示能力,从而影响其在各种任务中的性能。

解决这些问题对于确保 Transformer 模型在不同应用中的可靠性和有效性至关重要。

\section*{2. 自注意力的控制框架}

考虑第 $\ell$ 层的值矩阵:
\[
\mathbf{V}^{\ell} := 
\begin{bmatrix}
\mathbf{v}^{\ell}(1), \cdots, \mathbf{v}^{\ell}(N)
\end{bmatrix}^\top, \quad 
\mathbf{V}^{\ell} \in \mathbb{R}^{N \times D},
\]
如第 1.1 节所述。令 $\Omega \subset \mathbb{R}$,$\mathbf{v}(x, t) := [v_1(x, t), \cdots, v_D(x, t)]^\top$ 为一个实值向量函数,$\mathbf{v} : \Omega \times [0, \infty) \to \mathbb{R}^D$,且 $\mathbf{v} \in L^2(\Omega \times [0, \infty))$。假设值矩阵 $\mathbf{V}^\ell$ 离散化了空间和时间维度上的函数 $\mathbf{v}(x, t)$。在控制系统的上下文中,$\mathbf{v}(x)$ 可以被视为以下状态空间模型的状态信号:
\[
\frac{\partial \mathbf{v}(x, t)}{\partial t} = 
\int_\Omega (\mathbf{v}(y, t) - \mathbf{v}(x, t)) K(x, y, t) dy + \mathbf{z}(x, t),
\]
\[
\mathbf{v}(x, 0) = \mathbf{v}^0(x), \quad 
\mathbf{z}(x, t) = 0, \quad \forall x \in \Omega, \, \forall t \geq 0,
\]
其中 $\mathbf{z} \in L^2(\Omega \times [0, \infty))$ 是控制输入,$\mathbf{v}^0$ 是初始状态。函数 $K(x, y, t)$ 是核函数,用于捕获信号 $\mathbf{v}$ 在位置 $x, y$ 处以及时间 $t$ 时的接近程度。

这里,状态空间模型 (SSM) 是自治的,因为系统中没有引入任何控制输入或反馈。在本节中,我们展示了方程 (2) 描述的系统通过最小化信号的非局部全变差 (nonlocal total variation, Gilboa \& Osher, 2008) 来对信号进行平滑处理,从而在演化过程中丢失了详细信息。随后,我们表明,自注意力机制可以看作是这种动力学的离散化形式。最后,我们从理论上证明,方程 (2) 中的状态空间模型 (SSM) 容易受到输入扰动和表示能力坍塌的影响。

\section*{2.1 状态空间模型与非局部变分最小化的联系}

我们表明,以最小化以下非局部泛函为目标的\textbf{梯度流(gradient flow)}是方程 (2) 中描述的状态空间模型(SSM)的一种特殊情况:
\[
J(\mathbf{v}) = \frac{1}{2} \int_{\Omega \times \Omega} 
\|\mathbf{v}(x) - \mathbf{v}(y)\|_2^2 k(x, y) dx dy. \tag{3}
\]

这里,$J(\mathbf{v})$ 是空间维度上的非局部导数平方和:
\[
\partial_y \mathbf{v}(x) = \frac{\mathbf{v}(x) - \mathbf{v}(y)}{\sqrt{k(x, y)}}, 
\]
该非局部导数由 Gilboa 和 Osher (2008) 提出,用于表示信号 $\mathbf{v}$ 的非局部变化。而核函数 $k(x, y)$ 则捕捉了信号中位置 $x$ 和 $y$ 之间的接近程度。最小化 $J(\mathbf{v})$ 促进了 $\mathbf{v}$ 的平滑性,并抑制了信号中的高频成分。

对于 $\mathbf{v}$,其关于 $J$ 的梯度表示如下:
\[
\nabla_{\mathbf{v}} J(\mathbf{v}) = 
\begin{bmatrix}
\frac{\partial J}{\partial v_1}, \frac{\partial J}{\partial v_2}, \cdots, \frac{\partial J}{\partial v_D}
\end{bmatrix}^\top. \tag{4}
\]

如附录 B.10 所示,$J$ 关于 $\mathbf{v}_j$ 的\textbf{Fréchet 导数}为:
\[
\frac{\partial J}{\partial v_j} = 
\int_{\Omega} 
\left( v_j(x) - v_j(y) \right) \left( k(x, y) + k(y, x) \right) dy. \tag{5}
\]

将公式 (5) 中关于 $\partial J / \partial \mathbf{v}_j$ 的表达式代入公式 (4),可得到以下梯度流(gradient flow):
\[
\frac{\partial \mathbf{v}(x, t)}{\partial t} = -\nabla_{\mathbf{v}} J(\mathbf{v})
= \int_\Omega (\mathbf{v}(y, t) - \mathbf{v}(x, t))(k(x, y) + k(y, x)) dy, \tag{6}
\]
其中,方程 (2) 中的自治状态空间表示在 $K(x, y, t) := k(x, y) + k(y, x)$ 为对称且时间不变时,简化为上述形式。在这种情况下,模型减少了信号的非局部总方差,得到一个更平滑的解。然而,这使得模型容易在输出表示中出现秩坍塌问题。在第 2.2 节中,我们证明了无论 $K(x, y, t)$ 是否对称,模型都容易出现秩坍塌问题。

\subsection*{状态空间模型与自注意力机制的联系}
我们表明,状态空间模型(SSM)的离散化可以恢复自注意力机制。令 $\mathbf{q}, \mathbf{k} : \Omega \times [0, \infty) \to \mathbb{R}^{D_{qk}}$,且 $\mathbf{q}, \mathbf{k} \in L^2(\Omega \times [0, \infty))$ 为实值向量函数。类似于 $\mathbf{v}(x, t)$,我们可以在空间维度上对 $\mathbf{q}(x, t)$ 和 $\mathbf{k}(x, t)$ 进行离散化,得到第 $\ell$ 层的查询向量 $\mathbf{q}^\ell(1), \dots, \mathbf{q}^\ell(N) \in \mathbb{R}^{D_{qk}}$ 和键向量 $\mathbf{k}^\ell(1), \dots, \mathbf{k}^\ell(N) \in \mathbb{R}^{D_{qk}}$。定义接近核函数如下:
\[
K(x, y, t) := \frac{\exp(\mathbf{q}(x, t)^\top \mathbf{k}(y, t)/\sqrt{D_{qk}})}
{\int_\Omega \exp(\mathbf{q}(x, t)^\top \mathbf{k}(y', t)/\sqrt{D_{qk}}) dy'}.
\]

通过应用欧拉方法(Euler method)对方程 (2) 进行离散化,时间步长 $\Delta t(x) := 1$,系统的更新步骤变为:
\[
\mathbf{v}(x, t + 1) \approx 
\int_\Omega \frac{\exp(\mathbf{q}(x, t)^\top \mathbf{k}(y, t)/\sqrt{D_{qk}})}
{\int_\Omega \exp(\mathbf{q}(x, t)^\top \mathbf{k}(y', t)/\sqrt{D_{qk}}) dy'} 
\mathbf{v}(y, t) dy. \tag{7}
\]

使用蒙特卡罗方法(Monte-Carlo method, Metropolis \& Ulam, 1949)对空间维度上的积分进行近似,可以得到:
\[
\mathbf{v}^{\ell+1}(i) \approx \sum_{j=1}^N 
\text{softmax}\left( \frac{\mathbf{q}^\ell(i)^\top \mathbf{k}^\ell(j)}{\sqrt{D_{qk}}} \right) \mathbf{v}^\ell(j).
\]

这恢复了公式 (1) 中的 $\mathbf{u}^\ell(i)$,即第 $\ell$ 层中自注意力机制的输出令牌 $i$。由于自注意力机制是方程 (2) 中状态空间模型的离散化,它继承了该模型的特性,因此容易受到输入扰动和输出秩坍塌的影响。这些特性将在第 2.2 节中得到理论证明。

\section*{2.2 状态空间模型的稳定性与表示坍塌}

模型的鲁棒性指的是其在面临不确定或具有挑战性的场景(例如噪声数据、分布偏移或对抗性攻击)时,仍然能够维持高性能的能力。此外,鲁棒性还包括模型的稳定性,即当输入发生扰动时,模型的输出仍能保持相对不变。

对于状态空间模型(SSM)的理论分析,我们假设核函数 $K$ 是时间不变的,即 $K(x, y, t) = K(x, y)$。这种假设在 Transformer 的上下文中是实际的,特别是在深度 Transformer 模型中,注意力矩阵在初始几层后往往趋于相似。基于公式 (2) 的空间维度离散化,模型的状态空间表达式为:
\[
\frac{d\mathbf{v}(i, t)}{dt} = \sum_{j=1}^N (\mathbf{v}(j, t) - \mathbf{v}(i, t))K(i, j),
\]
其中 $i, j = 1, 2, \dots, N$。通过选择 $K(i, j) := \text{softmax}(\mathbf{q}(i)^\top \mathbf{k}(j) / \sqrt{D_{qk}})$,对应的矩阵表示可写为:
\[
\frac{d\mathbf{V}(t)}{dt} = K\mathbf{V}(t) - \mathbf{V}(t), \quad \mathbf{V}(0) = \mathbf{V}^0, \tag{8}
\]
其中 $K$ 是一个具有正元素的右随机矩阵。在 Transformer 的背景下,$K$ 是注意力矩阵,$\mathbf{V} = [\mathbf{v}^0(1), \dots, \mathbf{v}^0(N)]^\top$ 是第一层的值矩阵。引理 1 说明了公式 (8) 中 SSM 的稳定性及表示坍塌的特性。

\textbf{引理 1.} 给定 $\{\alpha_1, \alpha_2, \dots, \alpha_M\}$,其中 $M \leq N$,是 $K - I \in \mathbb{R}^{N \times N}$ 的复数谱。常微分方程 (8) 的解为:
\[
\mathbf{V}(t) = P\exp(Jt)P^{-1}\mathbf{V}^0, \tag{9}
\]
其中 $PJP^{-1}$ 是 $K - I$ 的 Jordan 分解,$P$ 是可逆矩阵,包含 $K - I$ 的广义特征向量,$J = \text{diag}(J_{\alpha_1, m_1}, J_{\alpha_2, m_2}, \dots, J_{\alpha_M, m_M})$ 是 $K - I$ 的 Jordan 形式,其中
\[
J_{\alpha_i, m_i} = 
\begin{bmatrix}
\alpha_i & 1 & \cdots & 0 \\
0 & \alpha_i & \cdots & 0 \\
\vdots & \vdots & \ddots & 1 \\
0 & 0 & \cdots & \alpha_i
\end{bmatrix} \in \mathbb{R}^{m_i \times m_i}, \quad i = 1, \dots, M,
\]
且 $\sum_{i=1}^M m_i = N$。

引理 1 的证明见附录 B.2。由于 $K$ 是一个正右随机矩阵,其最大的特征值 $\alpha_1$ 为 1,且 $\lvert \alpha_i \rvert < 1$(见 Bandiera et al., 2020 的定理 4.1),这意味着对于 $i = 2, \dots, M$,$\alpha_i \in [-1, 1]$。因此,矩阵 $K - I$ 的特征值为 $\alpha_1 - 1, \dots, \alpha_M - 1$,其中唯一最大的特征值为 0,其余特征值的实部位于 $[-2, 0]$。这导致稳态解的秩坍塌问题,具体描述见以下引理 2。

\textbf{引理 2.} 
\[
\lim_{t \to \infty} \mathbf{V}(t) = [c_{1,1}\mathbf{p}_1, \dots, c_{1, D_x}\mathbf{p}_1],
\]
其中 $\mathbf{p}_1$ 是 $K - I$ 的特征值 $(\alpha_1 - 1) = 0$ 对应的特征向量,$c_{1,1}, \dots, c_{1,D_x}$ 是常数。

\section*{3. 基于PID控制器的状态空间表示Transformer}

为了对抗因平滑性引起的信息丢失并增强模型稳定性,一个PID控制器被集成到状态空间表示中,形式如下:
\[
\frac{\partial \mathbf{v}(x, t)}{\partial t} =
\int_\Omega (\mathbf{v}(y, t) - \mathbf{v}(x, t)) K(x, y, t) dy + \mathbf{z}(x, t),
\]
\[
\mathbf{z}(x, t) = \lambda_P e(x, t) + \lambda_I \int_0^t e(x, \tau) d\tau + \lambda_D \frac{d e(x, t)}{d t},
\]
\[
\mathbf{v}(x, 0) = \mathbf{v}^0(x), \quad \mathbf{z}(x, 0) = 0. \tag{10}
\]

其中,正则化项定义为 $e(x, t) = f(x) - \mathbf{v}(x, t)$,它封装了随 $\mathbf{v}(x, t)$ 随时间变得更加平滑而丢失的信息。参考函数 $f(x)$ 表示一个包含输入原始详细信息的高频信号。在Transformer的上下文中,我们设置 $f(x) = \beta \mathbf{v}^0(i)$,其中 $\beta \in [0, 1]$ 是一个参数,用于控制我们希望保留的输入信号中的详细信息比例。

PID控制器的三个组成部分描述如下:
1. **P控制器**:该部分直接与正则化项 $e(x, t)$ 成比例。在大量信息丢失的情况下,控制输入 $\mathbf{z}(x, t)$ 应与比例因子 $\lambda_P$ 成比例,以重新引入丢失的信息。
2. **I控制器**:该部分累积所有过去的误差,由积分 $\lambda_I \int_0^t e(x, \tau) d\tau$ 给出。这一组件帮助重新引入被丢失的信息,同时稳定系统。
3. **D控制器**:该部分通过 $ \lambda_D \frac{d e(x, t)}{d t}$ 预测未来的信号变化速率,并增强控制系统的响应性和稳定性。

\subsection*{3.1 P和I控制器与不同优化方法的联系}

我们展示了加入PID控制器后的反馈状态空间模型(10)隐式最小化了以下泛函:
\[
E(\mathbf{v}, f) = J(\mathbf{v}) + G(\mathbf{v}, f),
\]
\[
J(\mathbf{v}) = \frac{1}{2} \int_{\Omega \times \Omega} \|\mathbf{v}(x) - \mathbf{v}(y)\|_2^2 k(x, y) dx dy,
\]
\[
G(\mathbf{v}, f) = \frac{\lambda}{2} \int_\Omega \|\mathbf{v}(x) - f(x)\|_2^2 dx,
\]
其中数据保真项 $G(\mathbf{v}, f)$ 通过惩罚信号与参考信号 $f$ 的偏差来保留信号中的显著信息。这进一步验证了系统(10)能够从参考信号 $f$ 中保留相关信息。

\textbf{P控制器下的状态空间模型作为梯度下降的优化方法}:泛函 $E(\mathbf{v}, f)$ 对 $\mathbf{v}$ 的梯度为:
\[
\nabla_{\mathbf{v}} E(\mathbf{v}, f) = \nabla_{\mathbf{v}} J(\mathbf{v}) + \lambda (\mathbf{v}(x) - f(x)),
\]
其梯度推导见附录 B.10。基于梯度下降方法,得到梯度流:
\[
\frac{\partial \mathbf{v}(x, t)}{\partial t} = -\nabla_{\mathbf{v}} E(\mathbf{v}, f),
\]
\[
\frac{\partial \mathbf{v}(x, t)}{\partial t} =
\int_\Omega (\mathbf{v}(y, t) - \mathbf{v}(x, t)) (k(x, y) + k(y, x)) dy
+ \lambda (\mathbf{v}(x, t) - f(x)). \tag{12}
\]

\subsection*{3.2 PID控制状态空间模型的稳定性与表示坍塌}

在本节中,我们的目标是展示以下几点:
(1) 集成比例控制器(P)部分能够增强模型的鲁棒性,使其能够对抗输入扰动,并缓解输出表示的秩坍塌问题;
(2) 在PID控制中添加微分控制器(D)部分可以进一步稳定系统,缓解 $\mathbf{V}(t)$ 的快速和不稳定变化;
(3) 最终,通过控制器的引入,PID控制状态空间模型(SSM)保证了系统的稳定性,使其对输入扰动具有鲁棒性。假设核函数 $K(x, y, t)$ 是时间不变的,我们在本节中对这一理论假设进行分析。

\subsubsection*{3.2.1 P控制状态空间模型的分析}

\textbf{P控制SSM的鲁棒性}:  
从状态空间模型 (10) 出发,通过选择 $\lambda_D = 0$ 并应用欧拉离散化(Euler discretization),P控制模型可表示为:
\[
\frac{d\mathbf{v}(i, t)}{dt} = \sum_{j=1}^N (\mathbf{v}(j, t) - \mathbf{v}(i, t)) K(i, j) 
+ \lambda_P (f(i) - \mathbf{v}(i, t)),
\]
其中 $i, j = 1, 2, \dots, N$,核函数 $K(i, j)$ 定义为:
\[
K(i, j) := \text{softmax}\left(\frac{\mathbf{q}(i)^\top \mathbf{k}(j)}{\sqrt{D_{qk}}}\right),
\]
对应的矩阵表示为:
\[
\frac{d\mathbf{V}(t)}{dt} = K\mathbf{V}(t) - (\lambda_P + 1)\mathbf{V}(t) + \lambda_P \mathbf{F}, \quad \mathbf{V}(0) = \mathbf{V}^0. \tag{18}
\]

这里 $\mathbf{F} = [f(1), \dots, f(N)]^\top$。引理 4 帮助我们分析 SSM (18) 的解的稳定性和表示坍塌问题。在这里,$K$ 的特征值实部位于 $[0, 1]$ 内,且 $1 + \lambda_P > 0$,因此矩阵 $(K - (\lambda_P + 1)I) \neq 0$。这意味着:
\[
\text{det}(K - (\lambda_P + 1)I) \neq 0,
\]
因此,以下结论成立。

\textbf{引理 4.}  
设 $B := K - (\lambda_P + 1)I \in \mathbb{R}^{N \times N}$,方程 (18) 的解为:
\[
\mathbf{V}(t) = \exp(Bt)\left(\mathbf{V}^0 + B^{-1}\mathbf{F}\right) - \lambda_P B^{-1}\mathbf{F}. \tag{19}
\]

如果 $B$ 的所有特征值实部为负,则:
\[
\lim_{t \to \infty} \mathbf{V}(t) = -\lambda_P B^{-1}\mathbf{F}.
\]

\textbf{引理 4 的证明}见附录 B.5。如第 2.2 节所述,由于 $K$ 的特征值 $\text{Re}(\alpha_i) \in [0, 1], i = 1, \dots, M$,矩阵 $B$ 的特征值实部必须位于 $[-2, -\lambda_P]$ 范围内。由于引理 4 的结果,系统状态 $\mathbf{V}(t)$ 会收敛到 $-\lambda_P B^{-1}\mathbf{F}$。因此,与初始状态 $\mathbf{V}^0$ 无关,系统会最终收敛到参考点 $f$ 的缩放版本 $\beta \mathbf{v}^0$。幸运的是,这种缩放稳定了最终解。

\subsubsection*{3.2.2 P控制SSM的分析}

\textbf{命题 1.}  
对于公式 (10) 中的比例系数 $\lambda_P > 0$ 和任意扰动 $\epsilon \in \mathbb{R}^{N \times D}$,$\|\epsilon\| \leq \epsilon_0$,通过将扰动加入到输入 $\mathbf{V}^0$ 中,系统的稳态解的变化是独立于 $\lambda_P$ 的,且其影响最多为:
\[
\beta \leq \frac{\delta}{\epsilon_0}. \tag{20}
\]

\textbf{证明}:命题 1 的证明见附录 B.6。此命题表明,我们可以选择超参数 $\beta$ 来控制输入扰动对输出的影响,使其尽可能小。

\textbf{P控制SSM的表示坍塌问题}:  
由于 $B^{-1}$ 是满秩矩阵($B$ 是非奇异矩阵),因此 $\text{rank}(-\lambda_P B^{-1}\mathbf{F}) = \text{rank}(\mathbf{F})$(见 Strang, 2006)。在 Transformer 的上下文中,当选择 $\mathbf{F} = \beta\mathbf{V}^0$ 时,稳态解的秩等于输入 $\mathbf{V}^0$ 的秩。这表明,P控制的动态特性与公式 (18) 的动态特性一致。

\subsubsection*{3.2.3 PID控制SSM的分析}

通过选择 $\lambda_D \frac{d e(x, t)}{d t} = -\lambda_D \frac{d\mathbf{v}(x, t)}{d t}$ 并令 $\lambda_P = \lambda_I = \lambda_D$,采用欧拉离散化方法对空间维度进行离散化,PID控制模型的矩阵形式为:
\[
\mathbf{V}'(t) = K\mathbf{V}(t) - (\lambda_P + 1)\mathbf{V}(t) + \lambda_P \mathbf{F} - \lambda_D \mathbf{V}'(t),
\]
\[
\mathbf{V}'(t) = \frac{1}{1 + \lambda_D} \left(K - (\lambda_P + 1)I\right)\mathbf{V}(t) + \frac{\lambda_P}{1 + \lambda_D}\mathbf{F}, \tag{21}
\]
其中 $\mathbf{V}(0) = \mathbf{V}^0$。方程 (21) 的解由以下引理提供。

\textbf{引理 5.}  
设 $B := K - (\lambda_P + 1)I \in \mathbb{R}^{N \times N}$,方程 (21) 的解为:
\[
\mathbf{V}(t) = \exp\left(\frac{1}{1 + \lambda_D}Bt\right)\left(\mathbf{V}^0 + B^{-1}\mathbf{F}\right) - \lambda_P B^{-1}\mathbf{F},
\]
\[
\lim_{t \to \infty} \mathbf{V}(t) = -\lambda_P B^{-1}\mathbf{F}.
\]

引理 5 的证明见附录 B.7。引入微分控制器(D部分)到P控制系统中不会改变系统的稳态解。此外,微分控制器的引入通过衰减特征值 $1/(1 + \lambda_D)$ 起到了进一步稳定系统的作用,从而缓解了 $\mathbf{V}(t)$ 的快速变化。

\textbf{命题 2.}  
对于任意 $\lambda_P, \lambda_I, \lambda_D > 0$,系统 (23) 是稳定的。

命题 2 的证明见附录 B.8。此命题表明,PID控制状态空间模型(SSM)在 (10) 中对于任意正值的 $\lambda_P, \lambda_I, \lambda_D$ 都是鲁棒且稳定的。

\subsection*{3.3 基于PID控制的Transformer}

通过欧拉离散化并令时间步长 $\Delta t = 1$,将 $t = 0$ 初始化为 $\mathbf{v}(0) = \mathbf{v}^0$,选择:
\[
K(x, y, t) := \frac{\exp(\mathbf{q}(x, t)^\top \mathbf{k}(y, t)/\sqrt{D_{qk}})}
{\int_\Omega \exp(\mathbf{q}(x, t)^\top \mathbf{k}(y', t)/\sqrt{D_{qk}}) dy'},
\]
PID控制SSM (10) 的更新步骤为:
\[
\mathbf{v}^{t+1}(x) \approx 
\int_\Omega \frac{\exp((\mathbf{q}^t(x)^\top \mathbf{k}^t(y))/\sqrt{D_{qk}})}
{\int_\Omega \exp((\mathbf{q}^t(x)^\top \mathbf{k}^t(y'))/\sqrt{D_{qk}}) dy'} \mathbf{v}^t(y) dy
+ \mathbf{v}^0(x) + \lambda_P e^t(x) + \lambda_I \sum_{m=1}^t e^m(x) + \lambda_D \frac{d e^t(x)}{d t}.
\]
使用蒙特卡罗方法对积分进行近似后,可进一步优化上述更新公式。

\section*{图1. 每层的PIDformer模型架构}

\noindent
通过在空间维度上对 $v^{\ell+1}(x), v^m(x), v^0(x)$ 离散化,并选择 $f(x) = v(x)$,我们得到以下新型PID注意力公式。它定义了第 $\ell$ 层的PID注意力输出。

\textbf{定义1} (PID控制Transformer,简称PIDformer)。对于每层 $\ell, \ell = 1, \dots, L$,给定一组键和值向量 $\{k^{\ell}(j), v^{\ell}(j)\}_{j=1}^N$,对于同一层中的每个查询向量 $q^{\ell}(i), i = 1, \dots, N$,在PID控制Transformer(PIDformer)的第 $\ell$ 层,自注意力单元通过以下公式计算查询 $q^{\ell}(i)$ 的对应输出向量 $u^{\ell}(i)$:

\[
u^{\ell}(i) = \sum_{j=1}^N \text{softmax}\left(\frac{q^{\ell}(i)^\top k^{\ell}(j)}{\sqrt{D_{qk}}}\right) v^{\ell}(j)
+ \lambda_P e^{\ell}(i) + \lambda_I \sum_{m=1}^\ell e^m(i) + \lambda_D \left(e^{\ell}(i) - e^{\ell-1}(i)\right), \tag{25}
\]

其中:
\[
e^{\ell} = v^0 - v^{\ell}, \quad v^0(1), \dots, v^0(N) \in \mathbb{R}^D
\]
是PIDformer第一层中的值向量。

由于PID注意力是受控状态空间模型(SSM)的离散化(参见公式(10)),它本质上是一种更加鲁棒的注意力机制。图1展示了PIDformer的架构。


\nocite{*}
\printbibliography[heading=bibintoc, title=\ebibname]

\appendix
%\appendixpage
\addappheadtotoc

\end{document}
