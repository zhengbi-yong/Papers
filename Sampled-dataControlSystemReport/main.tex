%!TEX program = xelatex
% 完整编译: xelatex -> biber/bibtex -> xelatex -> xelatex
\documentclass[lang=cn,a4paper,newtx]{elegantpaper}

\title{课程总结报告}
\author{雍征彼 \\ 北京理工大学}
\institute{}

% \version{0.11}
\date{\zhdate{2024/11/18}}

% 本文档命令
\usepackage{array}
\newcommand{\ccr}[1]{\makecell{{\color{#1}\rule{1cm}{1cm}}}}
\addbibresource[location=local]{reference.bib} % 参考文献,不要删除

\begin{document}

\maketitle

经过一学期复杂采样控制系统课程的学习,我深刻感受到了这一领域理论的深厚性和实践的挑战性。本课程基于陈通文教授和 Bruce Francis 教授的经典著作《Optimal Sampled-Data Control Systems》,结合史老师的讲解,让我在理论学习与应用实践中受益匪浅。以下是我对本课程学习的总结与心得体会。

\section{理论基础与核心概念的掌握}
课程伊始,我们学习了采样数据控制系统的基本概念及其在连续时间系统与离散时间系统之间的桥梁作用。这部分内容使我深入理解了采样器和零阶保持器的工作原理,学会了如何将连续时间系统离散化,并通过离散化模型分析和设计控制器。这种从连续到离散的转化方法,不仅是控制理论的核心问题之一,也为我理解数字控制系统的设计奠定了坚实基础。

通过系统建模与信号重构的理论,我认识到,采样数据系统能够在离散时间域内保留原连续系统的重要动态特性,从而为现代工业控制中的数字实现提供了科学依据。更重要的是,这种理论体系为分析混合信号系统提供了统一框架,使我对控制理论有了全新的认识。

\section{最优控制方法的深入理解}
课程的核心内容围绕采样数据系统中的最优控制展开,其中线性二次型调节器(LQR)、$H_2$控制和$H_\infty$控制等理论方法让我印象深刻。这些经典控制策略不仅在连续时间系统中意义深远,在采样数据系统中的拓展应用也同样重要。通过性能指标的构建与优化(如能量最小化或跟踪误差最小化),我学会了如何设计鲁棒性强、效率高的控制器。

在解决典型轨迹追踪问题的过程中,我对理论知识的理解进一步加深。例如,$H_2$控制注重性能优化,强调系统能量的高效利用;而$H_\infty$控制则更关注系统在不确定性和扰动条件下的稳定性。这种理论与实际问题相结合的学习方式,使我能从多角度分析和解决复杂系统问题,也使我在思考问题时更具系统性。

\section{跨学科知识的整合与应用}
采样数据控制系统涉及多个学科知识的交叉与融合,包括自动控制理论、信号处理、数学优化等。在学习系统稳定性和最优控制算法时,我深刻体会到矩阵论、离散傅里叶变换和复变函数等数学工具的强大威力。这种跨学科知识的整合,使我在理解理论时倍感挑战,但也激发了我主动查阅文献,巩固基础知识的学习热情。

模型预测控制(MPC)的设计问题是课程的另一亮点。通过学习其基本原理和设计方法,我对MPC在多约束条件下的动态优化能力有了更深理解。特别是在多个工业案例的分析中,我切实体会到了MPC的广泛应用性和强大适应性,这无疑提高了我的理论与实践能力。

\section{学术规范与研究方法的启发}
本课程不仅注重理论知识的讲解,还在学术写作方面给予了我很大启发。从陈通文教授与 Bruce Francis 教授的著作中,我学到了严谨的逻辑推导、规范的数学符号使用以及图表与公式的科学排版。学术写作不仅需要理论的扎实积累,还需要清晰的表达与规范的引用,这些细节体现了学术研究的专业性和对知识产权的尊重。

此外,通过课程的文献研读环节,我了解了控制领域中众多杰出学者的贡献与发展历程。例如,Kalman 教授在随机系统与状态估计领域的开创性工作和 George Zames 教授在鲁棒控制理论中的贡献,这些内容不仅拓展了我的学术视野,也让我对学术研究充满敬意。

\section{课程带来的启示与展望}
复杂采样控制系统课程的学习不仅帮助我掌握了控制系统设计的理论与方法,还让我对控制学科的历史与前沿有了更加深刻的理解。通过对经典理论的学习,我认识到控制系统作为一门工程学科的核心地位;通过对前沿问题的探索,我看到了控制领域未来发展的无限可能。

在未来的学习和研究中,我希望能够将这些理论应用到更具挑战性的实际问题中,特别是在与现代智能技术结合的控制系统设计中探索新的解决方法。尤其是在目前以经验学派为主的人工智能领域,控制理论在其中可能大有所为,作为一个已经发展了较久的学科,控制理论的方法和思想可能为人工智能领域提供新的思路和方法,并且可能能够系统化、理论化地解决一些人工智能领域中的问题。

这门课程不仅是一段学习旅程,更是一个启发我学术兴趣与研究热情的起点。我相信,这段经历将成为我未来研究道路上的重要基石。

\nocite{*}
\printbibliography[heading=bibintoc, title=\ebibname]

\appendix
%\appendixpage
\addappheadtotoc

\end{document}
